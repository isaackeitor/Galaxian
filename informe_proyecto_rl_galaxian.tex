\documentclass[conference]{IEEEtran}
\usepackage[utf8]{inputenc}
\usepackage[spanish]{babel}
\usepackage{graphicx}
\usepackage{amsmath}
\usepackage{booktabs}
\usepackage{multirow}
\usepackage{hyperref}
\usepackage{listings}
\usepackage{xcolor}

\lstset{
    basicstyle=\ttfamily\footnotesize,
    breaklines=true,
    frame=single,
    language=Python,
    showstringspaces=false
}

\title{Entrenamiento y Optimización de Agentes de Aprendizaje por Refuerzo Profundo para Atari Galaxian}

\author{
\IEEEauthorblockN{Josué Isaac Morales González}
\IEEEauthorblockA{Carné: 21116\\
Universidad del Valle de Guatemala\\
Proyecto Final - Reinforcement Learning\\
Deep Q-Networks, A2C y Dueling DDQN con PER\\
Entorno: ALE/Galaxian-v5 (Gymnasium)\\
Fecha: 21 de noviembre de 2025}
}

\begin{document}

\maketitle

\begin{abstract}
Este proyecto implementa y evalúa tres algoritmos de Aprendizaje por Refuerzo Profundo (Deep RL) para dominar el juego Atari Galaxian: DQN (Deep Q-Network), A2C (Advantage Actor-Critic) y Dueling Double DQN con Prioritized Experience Replay (PER). Se realizó una evaluación exhaustiva de 12 modelos entrenados, identificando problemas de sobreentrenamiento y desarrollando estrategias de optimización. Se implementó soporte para aceleración GPU en Apple Silicon (MPS), sistemas de continuación de entrenamiento desde checkpoints, y ajuste fino de hiperparámetros. Los resultados demuestran que el modelo DQN entrenado con 2,500 episodios alcanza el rendimiento óptimo (2,554 puntos promedio), mientras que el entrenamiento extendido causa degradación significativa (-54.4\% en 6,500 episodios). El modelo Dueling DDQN+PER, tras 27,000 episodios, muestra mejora sostenida de 800 a 1,800 puntos, validando la robustez de la arquitectura avanzada.
\end{abstract}

\section{Introducción}

El Aprendizaje por Refuerzo Profundo (Deep RL) ha revolucionado la capacidad de los agentes artificiales para dominar tareas complejas, desde juegos Atari hasta control robótico \cite{mnih2015human}. Este proyecto se enfoca en el juego \textit{Galaxian}, un shooter arcade que requiere toma de decisiones secuenciales en tiempo real bajo incertidumbre.

\subsection{Objetivos}
\begin{enumerate}
    \item Implementar desde cero tres algoritmos de Deep RL: DQN, A2C y Dueling DDQN+PER
    \item Evaluar sistemáticamente modelos entrenados con diferentes duraciones
    \item Identificar y resolver problemas de sobreentrenamiento
    \item Optimizar rendimiento computacional (GPU Apple Silicon)
    \item Desarrollar estrategias de continuación de entrenamiento
\end{enumerate}

\subsection{Contribuciones}
\begin{itemize}
    \item Evaluación exhaustiva de 12 modelos con 110 ejecuciones totales
    \item Identificación de punto óptimo de entrenamiento (2,500 episodios)
    \item Soporte MPS para aceleración en GPU Apple Silicon
    \item Sistema de continuación de entrenamiento con checkpoint loading
    \item Estrategia de aceleración para arquitecturas avanzadas
\end{itemize}

\section{Metodología}

\subsection{Entorno y Preprocesamiento}

\textbf{Entorno:} ALE/Galaxian-v5 (Gymnasium)\\
\textbf{Observaciones:} RGB frames (210 $\times$ 160 $\times$ 3)\\
\textbf{Acciones:} 6 acciones discretas (minimal action set)

\textbf{Pipeline de preprocesamiento:}
\begin{enumerate}
    \item \textit{Frame skipping}: 4 frames (ejecuta acción, observa cada 4 frames)
    \item \textit{Conversión a escala de grises}: RGB → Gris
    \item \textit{Redimensionamiento}: 210$\times$160 → 84$\times$84
    \item \textit{Frame stacking}: Apilar últimos 4 frames
    \item \textit{Normalización}: [0, 255] → [0, 1]
\end{enumerate}

Observación final: $(4, 84, 84)$ tensor (4 frames, canal-primero)

\subsection{Arquitecturas Implementadas}

\subsubsection{DQN (Deep Q-Network)}
Arquitectura Nature DQN \cite{mnih2015human}:
\begin{itemize}
    \item \textbf{Backbone CNN:}
    \begin{itemize}
        \item Conv1: 32 filtros, kernel 8$\times$8, stride 4
        \item Conv2: 64 filtros, kernel 4$\times$4, stride 2
        \item Conv3: 64 filtros, kernel 3$\times$3, stride 1
    \end{itemize}
    \item \textbf{Fully Connected Head:}
    \begin{itemize}
        \item FC1: 3,136 → 512 + ReLU
        \item FC2: 512 → 6 (Q-values)
    \end{itemize}
    \item \textbf{Técnicas:} Experience Replay (100K), Target Network
\end{itemize}

\textbf{Hiperparámetros DQN:}
\begin{itemize}
    \item Buffer: 100,000 transiciones
    \item Batch: 32
    \item Learning rate: $1 \times 10^{-4}$ (Adam)
    \item Discount factor ($\gamma$): 0.99
    \item Epsilon decay: 1.0 → 0.1 (300 episodios)
    \item Target update: cada 1,000 pasos
\end{itemize}

\subsubsection{A2C (Advantage Actor-Critic)}
Método de gradiente de política con baseline:
\begin{itemize}
    \item \textbf{Shared backbone:} Misma CNN que DQN
    \item \textbf{Actor head:} FC → softmax (logits de acciones)
    \item \textbf{Critic head:} FC → valor de estado $V(s)$
    \item \textbf{Rollouts:} n-step returns (n=5)
    \item \textbf{Advantage:} Generalized Advantage Estimation (GAE-$\lambda$, $\lambda$=0.95)
\end{itemize}

\textbf{Hiperparámetros A2C:}
\begin{itemize}
    \item Rollout length: 5 pasos
    \item Learning rate: $2.5 \times 10^{-4}$ (RMSprop)
    \item GAE $\lambda$: 0.95
    \item Entropy coefficient: 0.01
    \item Value loss coefficient: 0.5
\end{itemize}

\subsubsection{Dueling Double DQN con PER}
Combina tres mejoras sobre DQN:
\begin{itemize}
    \item \textbf{Dueling Architecture:} Separa streams de valor y ventaja
    \begin{equation}
    Q(s,a) = V(s) + \left(A(s,a) - \frac{1}{|A|}\sum_{a'}A(s,a')\right)
    \end{equation}
    \item \textbf{Double Q-Learning:} Selección de acción (red online) separada de evaluación (red target)
    \item \textbf{Prioritized Experience Replay:} Muestreo por prioridad basado en TD-error
    \begin{equation}
    P(i) = \frac{p_i^\alpha}{\sum_k p_k^\alpha}, \quad p_i = |\delta_i| + \epsilon
    \end{equation}
    Importance Sampling weights:
    \begin{equation}
    w_i = \left(\frac{1}{N} \cdot \frac{1}{P(i)}\right)^\beta
    \end{equation}
\end{itemize}

\textbf{Hiperparámetros DDQN+PER:}
\begin{itemize}
    \item Buffer: 100,000 (segment trees para O(log n) sampling)
    \item Batch: 32
    \item Learning rate: $1 \times 10^{-4}$ (Adam)
    \item PER $\alpha$: 0.6 (prioritization exponent)
    \item PER $\beta$: 0.4 → 1.0 (annealing)
    \item PER $\epsilon$: $1 \times 10^{-6}$
\end{itemize}

\section{Resultados Experimentales}

\subsection{Protocolo de Evaluación}
\begin{itemize}
    \item \textbf{Modelos evaluados:} 12 (11 DQN + 1 DDQN+PER)
    \item \textbf{Episodios por modelo:} 10 (total: 110 ejecuciones)
    \item \textbf{Política:} Greedy pura ($\epsilon=0$)
    \item \textbf{Métricas:} Promedio, máximo, mínimo, mediana
\end{itemize}

\subsection{Ranking de Modelos}

\begin{table}[h]
\centering
\caption{Top 5 Modelos por Rendimiento Promedio}
\label{tab:ranking}
\begin{tabular}{@{}clrrr@{}}
\toprule
\textbf{Rank} & \textbf{Modelo} & \textbf{Eps} & \textbf{Prom.} & \textbf{Máx.} \\ \midrule
\#1 & DQN & 2,500 & \textbf{2,554} & \textbf{4,630} \\
\#2 & DQN & 2,000 & 2,388 & - \\
\#3 & DQN & 4,500 & 2,259 & - \\
\#4 & DQN & 4,000 & $\sim$2,100 & - \\
\#5 & DQN & 3,500 & $\sim$2,000 & - \\
\midrule
\#11 & DQN & 6,500 & 1,165 & - \\
\#12 & DDQN+PER & 24,600 & 847 & 1,350 \\
\bottomrule
\end{tabular}
\end{table}

\subsection{Análisis de Sobreentrenamiento}

\textbf{Modelo óptimo:} DQN 2,500 episodios (2,554 puntos, max 4,630)

La Figura \ref{fig:dqn_optimo} muestra la curva de aprendizaje del mejor modelo, con mejora sostenida desde 500 hasta alcanzar su pico alrededor del episodio 2,000-2,500.

\begin{figure}[h]
\centering
\includegraphics[width=\columnwidth]{dqn_optimo.png}
\caption{Curva de aprendizaje DQN óptimo (2,500 episodios). Media móvil alcanza $\sim$3,100 puntos, con picos individuales hasta 6,800 puntos. Modelo con mejor rendimiento en evaluación (2,554 pts promedio).}
\label{fig:dqn_optimo}
\end{figure}

\textbf{Evidencia de degradación:}
\begin{itemize}
    \item DQN 5,000 ep: $\sim$1,850 pts (-27.6\%)
    \item DQN 6,000 ep: $\sim$1,600 pts (-37.4\%)
    \item DQN 6,500 ep: 1,165 pts (-54.4\%) \textit{Colapso catastrófico}
    \item DQN 7,000 ep: 1,748 pts (-31.6\%) \textit{Recuperación parcial}
\end{itemize}

\textbf{Fases de entrenamiento identificadas:}
\begin{enumerate}
    \item \textit{Aprendizaje} (0-2,500): Mejora progresiva, pico en 2,500
    \item \textit{Plateau} (2,500-5,000): Fluctuaciones, ligera degradación
    \item \textit{Degradación severa} (5,000-7,000): Overfitting evidente
\end{enumerate}

\subsection{Resultados DDQN+PER Inicial}

El modelo DDQN+PER entrenado 24,600 episodios mostró el \textbf{peor rendimiento}:
\begin{itemize}
    \item Promedio: 847 puntos (-66.8\% vs óptimo)
    \item Rango: 240-1,350 puntos (alta variabilidad)
    \item Hipótesis: Sobreentrenamiento extremo, hiperparámetros no optimizados
\end{itemize}

Sin embargo, un análisis posterior del entrenamiento inicial (primeros 9,500 episodios) reveló un comportamiento estable sin colapsos, como se muestra en la Figura \ref{fig:ddqn_inicial}.

\begin{figure}[h]
\centering
\includegraphics[width=\columnwidth]{ddqn_per_inicial.png}
\caption{Curva de aprendizaje DDQN+PER inicial (0-9,500 episodios). Media móvil estable alrededor de 600 puntos sin degradación.}
\label{fig:ddqn_inicial}
\end{figure}

\section{Optimizaciones Implementadas}

\subsection{Aceleración GPU - Apple Silicon (MPS)}

PyTorch 1.12+ soporta Metal Performance Shaders (MPS) para GPUs Apple Silicon. Implementamos detección automática de dispositivo:

\begin{lstlisting}
if torch.backends.mps.is_available():
    device = torch.device("mps")
elif torch.cuda.is_available():
    device = torch.device("cuda")
else:
    device = torch.device("cpu")
\end{lstlisting}

\textbf{Ganancia de velocidad:} $\sim$3-5x vs CPU para DDQN+PER (operaciones tensoriales intensivas)

\subsection{Sistema de Continuación de Entrenamiento}

Desarrollamos notebooks para continuar entrenamiento desde checkpoints:

\textbf{Componentes guardados en checkpoint:}
\begin{itemize}
    \item Pesos de red Q y red target (state\_dict)
    \item Estado del optimizador
    \item Episodio actual y pasos globales
    \item Historial de recompensas
    \item Metadata (input\_shape, n\_actions)
\end{itemize}

\textbf{Estrategia de ajuste de hiperparámetros:}
\begin{enumerate}
    \item Cargar checkpoint
    \item \textit{Mantener} hiperparámetros originales (buffer, batch, LR, gamma)
    \item \textit{Reducir} epsilon (0.15 → 0.05, ya exploró)
    \item Crear nuevo buffer (se llena durante entrenamiento)
\end{enumerate}

\textbf{Lección aprendida:} Modificar hiperparámetros durante continuación causó colapsos (DQN: 3,000 → 1,800 pts). Solución: usar valores originales probados.

\subsection{Entrenamiento Continuo DDQN+PER}

Reentrenamos DDQN+PER desde episodio 1,000 hasta 27,000:

\textbf{Resultados:}
\begin{itemize}
    \item Eps 0-20,000: Media móvil estable $\sim$600-800 pts
    \item Eps 20,000-27,000: \textbf{Explosión de mejora} 800 → 1,800 pts (+125\%)
    \item Picos máximos: 5,000-6,000 puntos
    \item \textbf{Conclusión:} Arquitectura robusta, mejora sostenida sin colapso
\end{itemize}

La Figura \ref{fig:ddqn_continuado} muestra la evolución completa del entrenamiento, evidenciando la mejora sostenida sin colapsos característicos del DQN simple.

\begin{figure}[h]
\centering
\includegraphics[width=\columnwidth]{ddqn_per_continuado.png}
\caption{Entrenamiento DDQN+PER continuo (0-26,000 episodios). La media móvil muestra mejora sostenida de 600 a 1,800 puntos después del episodio 20,000, alcanzando picos de 6,000 puntos sin degradación.}
\label{fig:ddqn_continuado}
\end{figure}

\subsection{Estrategia de Aceleración}

Para acelerar la mejora observada en DDQN+PER (episodios 27,000+):

\begin{table}[h]
\centering
\caption{Hiperparámetros Acelerados para DDQN+PER}
\label{tab:aceleracion}
\begin{tabular}{@{}lrrr@{}}
\toprule
\textbf{Parámetro} & \textbf{Original} & \textbf{Acelerado} & \textbf{Cambio} \\ \midrule
Learning rate & $1e-4$ & $2.5e-4$ & +150\% \\
Batch size & 32 & 64 & +100\% \\
Buffer capacity & 100K & 80K & -20\% \\
Epsilon inicial & 0.15 & 0.10 & -33\% \\
Epsilon final & 0.05 & 0.02 & -60\% \\
PER $\alpha$ & 0.6 & 0.7 & +17\% \\
PER $\beta$ inicial & 0.4 & 0.5 & +25\% \\
Target update & 1,000 & 1,200 & +20\% \\
\bottomrule
\end{tabular}
\end{table}

\textbf{Justificación:}
\begin{itemize}
    \item \textit{LR alto:} Aprende 2.5x más rápido (modelo ya estable)
    \item \textit{Batch grande:} Gradientes más estables, menos ruido
    \item \textit{Buffer pequeño:} Olvida experiencias antiguas, enfoca en lo reciente
    \item \textit{Epsilon bajo:} 90\% explotación (modelo ya sabe jugar)
    \item \textit{PER $\alpha$ alto:} Prioriza más errores grandes (aprendizaje eficiente)
\end{itemize}

\textbf{Proyección:} Mejora esperada de 1,800 → 2,800 pts en 3,000 episodios ($\sim$2x velocidad original)

\section{Problemas Técnicos Resueltos}

\subsection{Compatibilidad Bilingüe de Checkpoints}

\textbf{Problema:} Modelos iniciales usaban claves en español (\texttt{'red\_q'}, \texttt{'episodio'}) vs inglés (\texttt{'q\_network\_state'}, \texttt{'episode'})

\textbf{Solución:} Detección automática de idioma en cargado de checkpoints:
\begin{lstlisting}
if 'red_q' in checkpoint:
    # Checkpoint en espanol
    model.load_state_dict(checkpoint['red_q'])
elif 'q_network_state' in checkpoint:
    # Checkpoint en ingles
    model.load_state_dict(checkpoint['q_network_state'])
\end{lstlisting}

\subsection{Arquitectura Dueling Incompatible}

\textbf{Problema:} DDQN+PER usa streams separados (\texttt{stream\_valor}, \texttt{stream\_ventaja}) incompatibles con DQN

\textbf{Solución:} Implementación de clase separada \texttt{DuelingDDQN} con agregación correcta:
\begin{lstlisting}
V = self.value_stream(features)  # (B, 1)
A = self.advantage_stream(features)  # (B, 6)
Q = V + (A - A.mean(dim=1, keepdim=True))
\end{lstlisting}

\subsection{Manejo de Layouts de Observación}

\textbf{Problema:} Wrappers pueden retornar $(C, H, W)$ o $(H, W, C)$

\textbf{Solución:} Detección automática y permutación en \texttt{forward()}:
\begin{lstlisting}
if x.shape[1] != expected_c and x.shape[-1] == expected_c:
    x = x.permute(0, 3, 1, 2)  # NHWC -> NCHW
\end{lstlisting}

\section{Conclusiones y Trabajo Futuro}

\subsection{Conclusiones}

\begin{enumerate}
    \item \textbf{Punto óptimo de entrenamiento:} 2,500 episodios para DQN simple (2,554 pts promedio, máx 4,630)

    \item \textbf{Sobreentrenamiento severo:} Degradación de hasta -54.4\% después de 5,000 episodios. El entrenamiento extendido NO mejora rendimiento.

    \item \textbf{Arquitecturas avanzadas requieren más tiempo:} DDQN+PER fracasó inicialmente (847 pts @ 24K eps) pero mostró mejora sostenida con entrenamiento continuo (1,800 pts @ 27K eps).

    \item \textbf{Robustez de Dueling DDQN+PER:} Sin colapsos observados en 27,000 episodios, mejora sostenida +125\% (800→1,800).

    \item \textbf{Importancia de hiperparámetros:} Modificaciones durante continuación causaron colapsos. Valores originales probados son críticos.

    \item \textbf{GPU Apple Silicon efectiva:} MPS proporciona 3-5x aceleración vs CPU para redes convolucionales profundas.
\end{enumerate}

\subsection{Trabajo Futuro}

\begin{enumerate}
    \item \textbf{Early Stopping:} Implementar validación periódica (cada 500 eps) con criterio de parada automático

    \item \textbf{Regularización:} Explorar dropout, weight decay, layer normalization para prevenir overfitting

    \item \textbf{Ensemble Methods:} Combinar múltiples modelos (DQN 2000, 2500, 4500) para reducir varianza

    \item \textbf{Análisis de Representaciones:} Visualizar activaciones de CNN en modelos óptimos vs sobreentrenados

    \item \textbf{Transfer Learning:} Evaluar si características aprendidas en Galaxian transfieren a otros shooters (Space Invaders, Breakout)

    \item \textbf{Curriculum Learning:} Entrenar con dificultad progresiva (ajustar velocidad de enemigos)

    \item \textbf{Comparación con Rainbow DQN:} Implementar combinación de todas las mejoras (Dueling + Double + PER + Distributional + Noisy Nets + Multi-step)
\end{enumerate}

\subsection{Recomendaciones para Producción}

\begin{itemize}
    \item \textbf{Modelo recomendado:} DQN 2,500 episodios
    \item \textbf{Evaluación:} $\epsilon = 0$ (greedy puro)
    \item \textbf{Monitoreo:} Registrar media móvil (100 eps) para detectar degradación
    \item \textbf{Límite de entrenamiento:} No exceder 3,000 episodios sin validación
    \item \textbf{Arquitecturas avanzadas:} DDQN+PER viable con 25,000+ episodios y monitoreo continuo
\end{itemize}

\section{Código y Reproducibilidad}

\textbf{Repositorio:} Disponible en Lab-10 (código fuente completo)

\textbf{Notebooks principales:}
\begin{itemize}
    \item \texttt{ProyectoFinal\_RL\_MPS.ipynb}: Implementaciones base (DQN, A2C, DDQN+PER) con soporte MPS
    \item \texttt{DQN\_Continuar\_Entrenamiento.ipynb}: Continuación DQN desde checkpoints
    \item \texttt{DDQN\_Continuar\_Entrenamiento.ipynb}: Continuación DDQN+PER acelerada
\end{itemize}

\textbf{Scripts de evaluación:}
\begin{itemize}
    \item \texttt{play\_dqn.py}: Testing de modelos DQN
    \item \texttt{play\_ddqn\_per.py}: Testing de modelos DDQN+PER
    \item \texttt{dqn\_model.py}, \texttt{ddqn\_per\_model.py}: Arquitecturas
    \item \texttt{dqn\_policy.py}, \texttt{ddqn\_per\_policy.py}: Wrappers de política
\end{itemize}

\textbf{Dependencias principales:}
\begin{itemize}
    \item PyTorch 2.0+ (con soporte MPS)
    \item Gymnasium 0.29+
    \item ALE-Py (Atari Learning Environment)
    \item NumPy, Matplotlib, OpenCV
\end{itemize}

\textbf{Comando de instalación:}
\begin{lstlisting}[language=bash]
pip install gymnasium[atari] ale-py autorom \
    torch torchvision imageio imageio-ffmpeg
AutoROM --accept-license
\end{lstlisting}

\section*{Agradecimientos}

Este proyecto fue desarrollado como parte del curso de Reinforcement Learning. Agradecemos a la comunidad de Gymnasium y OpenAI por proporcionar el framework de entrenamiento, y a DeepMind por la arquitectura Nature DQN que sirvió de base.

\begin{thebibliography}{9}

\bibitem{mnih2015human}
Mnih, V., et al. (2015).
\textit{Human-level control through deep reinforcement learning.}
Nature, 518(7540), 529-533.

\bibitem{wang2016dueling}
Wang, Z., et al. (2016).
\textit{Dueling Network Architectures for Deep Reinforcement Learning.}
ICML.

\bibitem{schaul2015prioritized}
Schaul, T., et al. (2015).
\textit{Prioritized Experience Replay.}
arXiv:1511.05952.

\bibitem{vanhassselt2016deep}
van Hasselt, H., et al. (2016).
\textit{Deep Reinforcement Learning with Double Q-learning.}
AAAI.

\bibitem{mnih2016asynchronous}
Mnih, V., et al. (2016).
\textit{Asynchronous Methods for Deep Reinforcement Learning.}
ICML.

\end{thebibliography}

\end{document}
